Base de datos

Crear una base de datos llamada cine y dar permisos al usuario por defecto \begin{DoxyVerb}CREATE DATABASE cine;
CREATE USER IF NOT EXISTS 'spq'@'localhost' IDENTIFIED BY 'spq';
GRANT ALL ON cine.* TO 'spq'@'localhost';
\end{DoxyVerb}
 La configuración por defecto para la base de datos y los usuarios puede ser actualizada en el fichero resources/datanucleus.\+properties.

Creación de las tablas

Para la creación de las tablas se debe ejecutar el comando de maven \begin{DoxyVerb}mvn compile datanucleus:schema-create
\end{DoxyVerb}
 Enhance del proyecto \begin{DoxyVerb}mvn datanucleus:enhance
\end{DoxyVerb}
 Datos de prueba

Se pueden introducir datos de prueba en la aplicación utilizando el comando de maven \begin{DoxyVerb}mvn exec::java -Pdatos
\end{DoxyVerb}
 Inicio del servidor

El servidor REST de la aplicación se lanza utilizando el comando \begin{DoxyVerb}mvn exec::java
\end{DoxyVerb}
 Si el servidor ha sido iniciado correctamente se pueden obtener los datos de prueba accediendo con el navegador a la URL {\texttt{ http\+://localhost\+:8080/myapp/films}}.

Inicio de la aplicación cliente

La aplicación cliente puede iniciarse usando el comando \begin{DoxyVerb}mvn exec::java -Pcliente
\end{DoxyVerb}
 Para comprobar los teses de integracion \begin{DoxyVerb}mvn verify -P integration
\end{DoxyVerb}
 Para comprobar los teses de la ventana \begin{DoxyVerb}mvn verify -P gui
\end{DoxyVerb}
 \DoxyHorRuler{0}
 